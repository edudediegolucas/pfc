\chapter*{\textbf{Abstract}}

Medical Implant Devices (MID) are small devices which are placed inside the human body or partially inside it with the aim of correcting organic dysfunctions or monitorizing physiological parametres. Due to its position inside the human body, the MID are usually equipped with a communication system that allows to config the characteristics of the MID and to recover monitorized data from an external device.

The first MID communication systems were made with inductive technologies. The several limitations of these technologies (low transmission velocity, short distance to maintain the link, etc.) added to a bigger growth of electrical materials and components, were the seed to design new technologies based on radiofrequency (RF). In these new technologies, the MID communication system's antenna has a essential function because it has to fit with the MID properties: small, efficient, low power and of course, it must be embedded to the device that is going to be implanted. The \textit{microstrip antennas} have some characteristics in order to do them appropriate for this kind of RF technologies. There are a lot of different microstrip antenna designs due to there is not a unique MID model.

This End-Of-Degree project consists in studying some analysis, designs and properties of a microstrip antenna based in a design made in a scientific article for MID. Through a parametric study and computer simulations, results are obtained with graphs and illustrations. The microstrip simulated antenna fulfills the objective function as a tool for DMI communication system.

