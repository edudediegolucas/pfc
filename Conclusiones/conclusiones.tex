\chapter{\textbf{Conclusiones y líneas futuras}}\label{conclusiones}

En este Proyecto Fin de Carrera se ha estudiado, analizado y diseñado una antena microstrip en forma de serpentina para sistemas de comunicación de los Dispositivos Médicos Implantables (DMI). Los pasos que se han realizado han sido los siguientes:

\begin{itemize}
    \item Se ha efectuado una revisión de artículos de la literatura referente a las antenas microstrip en DMI. Los artículos destacados son investigaciones recientes de distintos tipos de antenas para diferentes propósitos (sección \ref{sec:aplicacionesdmi} del capítulo \ref{ch:contexto}).
    \item Se ha realizado un análisis de la antena propuesta por Soontornpipit en \cite{soont}. A partir de ese diseño, se ha analizado la eficiencia de dicha antena en un entorno en espacio libre y en un entorno de tejido muscular a través del software CST (sección \ref{sec:original} del capítulo \ref{ch:simulaciones}).
    \item A continuación, se ha realizado un estudio paramétrico de las variables de la antena que ha dado paso a una modificación de las dimensiones del parche radiante para ajustar la frecuencia de resonancia a la banda MICS, mediante simulaciones en tejido muscular. La antena rediseñada con los mismos materiales que en \cite{soont} cumple con su función e incluso mejora algunos parámetros como el $S_{11}$ (sección \ref{sec:modificado} del capítulo \ref{ch:simulaciones}).
    \item Se ha estudiado la posibilidad de simular la antena modificada en un entorno de tres capas de tejido (piel, músculo y grasa). La antena responde bien a la simulación, aunque empeorando sus resultados respecto a la simulación únicamente en tejido múscular (subsección \ref{subsec:3capas} del capítulo \ref{ch:simulaciones}).
    \item Se ha propuesto otro diseño de la misma antena con distintos materiales a los propuestos en \cite{soont}. Esto ha implicado un nuevo estudio paramétrico para cambiar los parámetros del parche radiante y el grosor de los substratos. El objetivo de centrar la frecuencia de resonancia en la banda MICS en tejido muscular se ha conseguido en detrimento de la eficiencia con respecto al diseño con los mismos materiales del artículo (sección \ref{sec:optimizado} del capítulo \ref{ch:simulaciones}).
\end{itemize}


\section{Conclusiones}

Este Proyecto Fin de Carrera ha tenido como objetivo el estudio de una antena de tecnología microstrip para DMI. Podemos decir que este tipo de antenas son aptas para estos dispositivos en el entorno de tejido muscular que se ha analizado. Se puede afirmar por tanto, que las antenas microstrip son una posibilidad viable para los enlaces de comunicaciones de los DMI hoy en día, en lugar de los enlaces inductivos los cuales tienen importantes limitaciones como se ha mencionado a lo largo de este proyecto.

Tras presentar el diseño de una antena microstrip con parche radiante en forma de serpentina~\cite{soont}, se ha comprobado mediante simulaciones en el software \textit{CST Studio Suite} que la antena es viable para los DMI. Los diagramas de radiación presentados han demostrado que el diseño de antena microstrip elegido radia con un lóbulo de gran ancho de haz. Sin embargo, la antena simulada presenta baja eficiencia, presente en en todas las simulaciones debida seguramente a la adaptación de impedancias. Por otro lado, el estudio paramétrico que se ha realizado para encontrar un parámetro $S_{11}$ ajustado a la banda MICS ha sido satisfactorio, si bien los autores del artículo no consideran este diseño como adecuado para el parámetro $S_{11}$.

El último diseño propuesto, con distintos materiales al del artículo, indica que el prototipo de antena con parche radiante en forma de serpentina, puede ser implementado junto a DMI. Algunos artículos en la literatura como~\cite{requena} o~\cite{kim} prefieren el estudio del parche radiante en forma de espiral, argumentando que muestra mejor eficiencia que el parche en serpentina, pero en este proyecto se ha estudiado que el diseño en serpentina puede ser también una gran opción para futuros DMI.

A partir de los resultados de las simulaciones, podemos decir que existen diferencias notables entre el entorno de espacio libre y los entornos con tejidos humanos (músculo o tres capas: piel, grasa y músculo). La antena propuesta está diseñada para un entorno de tejido humano, en especial, el tejido muscular como hemos comprobado en las simulaciones. Hemos comprobado también que el cambio de materiales de la antena influye en el comportamiento de la misma. Los materiales originales, muestran mayor eficiencia en el parámetro $S_{11}$ y directividad que los materiales propuestos en la última sección~\ref{sec:optimizado}. Es posible, como hemos comentado, que cambiando las dimensiones de la antena se pueda adaptar mejor con los nuevos materiales a las condiciones de simulación. Hemos de mencionar además, la diferencia existente entre nuestras simulaciones y las simulaciones de los autores: como dijimos, los autores de~\cite{soont} no indican qué software están utilizando, por lo que los resultados obtenidos en este proyecto no se ajustan a las condiciones computacionales del artículo.

Es visible comprobar las limitaciones de este tipo de estudios únicamente basados en simulaciones. La antena propuesta en este proyecto podría ser fabricada para analizar su comportamiento en pruebas reales y así puntualizar los resultados obtenidos en las simulaciones. Lo recogido en este proyecto es parte de lo que podría ser una investigación completa en laboratorio. Así por ejemplo, se podría medir el índice SAR de esta antena y comprobar otra serie de medidas para su futura adaptación a DMI. Es posible que el prototipo fabricado a partir de este diseño, se alejase ligeramente de los resultados numéricos, debido seguramente a impurezas en la fabricación, pero son puntos a tener en cuenta cuando se trabaja en este campo.

A lo largo del Proyecto Fin de Carrera se ha profundizado en el campo de las antenas microstrip. Como primera aproximación que se ha tenido a este tipo de materia y tecnologías hemos alcanzado unos niveles aceptables de aprendizaje. Algunos artículos relacionados con este campo se han analizado en la sección~\ref{sec:aplicacionesdmi}, lo que ha supuesto un siguiente estudio de las distintas propiedades electromagnéticas de los tejidos humanos. Las propiedades mencionadas han ayudado de forma notable a comprender el comportamiento de las simulaciones realizadas en los distintos medios propuestos. El estudio de las bandas MICS e ISM, sobre todo de la primera, ha enfocado nuestro estudio paramétrico del parámetro $S_{11}$ para las simulaciones.

Otro punto a destacar es la pequeña introducción realizada al programa CST Studio Suite 2010 para la simulación de antenas. Es un programa que no se utiliza habitualmente en los cursos académicos de la universidad por lo que ha exigido un esfuerzo mayor para este proyecto. El aprendizaje para manejarlo y presentar los datos forma parte del propio proyecto.


\section{Líneas futuras}

Las distintas líneas futuras o futuras investigaciones que se han ido mencionando a lo largo de este Proyecto Fin de Carrera quedan recogidas aquí:

\begin{itemize}
    \item La fabricación del último diseño realizado con distintos materiales a los del artículo y comprobar su eficiencia en pruebas de laboratorio.
    \item Incorporar la antena fabricada en un prototipo de sistema de comunicaciones, como por ejemplo, con el chip de \textit{Microsemi} visto en la subsección \ref{subsec:comparacion} del capítulo~\ref{ch:cap3}~\cite{zarlink}.
    \item Mejorar el diseño de serpentina con otros materiales y/o dimensiones para que funcione correctamente en un entorno de tres capas (piel, músculo y grasa). Este entorno es el más parecido a la realidad por lo que encontrar un diseño que funcione en la banda MICS correctamente sería excepcional.
    \item Diseñar una antena aún más eficiente. En este punto puede entrar en juego remodelar la estrucutra entera de la antena, cambiar la forma del parche radiante o incluso encontrar un método de alimentación más eficiente.
\end{itemize}
