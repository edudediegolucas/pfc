\chapter{\textbf{Glosario}}

\begin{description}

    \item[ANSI]\mbox{}\\
    \textbf{American National Standards Institute.} Insitituto Nacional de Estándares Americano.

    \item[DBS]\mbox{}\\
    \textbf{Direct Broadcast Services.} Aplicación de telecomunicaciones que utiliza los satélites que orbitan alrededor de la Tierra.

    \item[DMI]\mbox{}\\
    \textbf{Dispositivo Médico Implantable.} Aparato utilizado en medicina para funciones terapéuticas o de monitorización.

    \item[DSSS]\mbox{}\\
    \textbf{Direct-sequence Spread Spectrum.} Espectro ensanchado por Secuencia Directa. Técnica utilizada en estándares de radiofrecuencia que consiste en aumentar el ancho de banda de la portadora por encima de los umbrales para reducir interferencias.

    \item[ERP]\mbox{}\\
    \textbf{Equivalent Radiated Power.} Potencia Radiada Equivalente.

    \item[ETSI]\mbox{}\\
    \textbf{European Telecommunications Standards Institute.} Instituto Europeo de Estándares en Telecomunicaciones.

    \item[FCC]\mbox{}\\
    \textbf{Federal Communications Commission.} Agencia independiente del gobierno de Estados Unidos que regula y estandariza sistemas de todo tipo.

    \item[FDTD]\mbox{}\\
    \textbf{Finite-Difference Time-Domain Method.} Método de las diferencias finitas en el dominio del tiempo. Utilizada por los modelos numéricos en el estudio de las propiedades electromagnéticas de distintos cuerpos.

    \item[FEM]\mbox{}\\
    \textbf{Finite Element Method.} Método de los elementos finitos. Utilizada por los modelos numéricos en el estudio de las propiedades electromagnéticas de distintos cuerpos.

    \item[FFT]\mbox{}\\
    \textbf{Fast Fourier Transform.} Transformada Rápida de Fourier.

    \item[FHSS]\mbox{}\\
    \textbf{Frequency-Hopping Spread Spectrum.} Espectro Ensanchado por Salto en Frecuencia. Técnica utilizada en estándares de radiofrecuencia que consiste en modificar la portadora entre varias frecuencias por medio de un código que conocen transmisor y receptor.

    \item[GPS]\mbox{}\\
    \textbf{Global Position System.} Sistema de Posicionamiento Global. Servicio que permite conocer la localización exacta en cualquier parte de la Tierra por medio de 24 satélites geoestacionarios.

    \item[Hz]\mbox{}\\
    \textbf{Hercio.} Unidad de medida que representa los ciclos por cada segundo de tiempo que oscila un objeto.

    \item[ICNIRP]\mbox{}\\
    \textbf{International Commission on Non-Ionizing Radiation Protection.} Comisión Internacional sobre la Protección de Radiaciones No-Ionizantes.

    \item[IEEE]\mbox{}\\
    \textbf{Institute of Electrical and Electronics Engineers.} Instituto de Ingenieros Eléctricos y Electrónicos.

    \item[ISM]\mbox{}\\
    \textbf{Industrial, Scientific and Medical band.} Banda del espectro electromagnético en torno a los 2.45 GHz utilizada para implantes y servicios médicos.

    \item[Kg]\mbox{}\\
    \textbf{Kilogramo.} Unidad de masa del Sistema Internacional.

    \item[METAIDS]\mbox{}\\
    \textbf{Meteorological Aids Service.} Servicio de Apoyo Meteorológico.

    \item[MICS]\mbox{}\\
    \textbf{Medical Implant Communication Service band.} Banda del espectro electromágnetico en torno a los 403 MHz utilizada para implantes y servicios médicos.

    \item[MID]\mbox{}\\
    \textbf{Medical Implant Device.} Dispositivo Médico Implantable o DMI.

    \item[RF]\mbox{}\\
    \textbf{Radio Frecuencia.} Serie de bandas del espectro electromagnético en las que son posibles la propagación de ondas electromagnéticas con información.

    \item[RT]\mbox{}\\
    \textbf{Ray Tracing.} Técnica de Trazado de Rayos. Utilizada por los modelos numéricos en el estudio de las propiedades electromagnéticas de distintos cuerpos.

    \item[$S_{11}$]\mbox{}\\
    \textbf{Parámetro de Scattering 1-1.} Describe el comportamiento de reflexión de potencia de un dispositivo que trabaja en frecuencias de microondas.

    \item[$S_{21}$]\mbox{}\\
    \textbf{Parámetro de Scattering 2-1.} Describe la ganancia de tensión en directa de un dispositivo que trabaja en frecuencias de microondas.

    \item[SAR]\mbox{}\\
    \textbf{Specific Absortion Rate.} Tasa de absorción específica. Índice que mide la cantidad de potencia por unidad de masa que es capaz de absorber un cuerpo a una cierta frecuencia de radiación.

    \item[TE]\mbox{}\\
    \textbf{Transversal Eléctrico.} Modo asociado a una onda electromagnética la cual no tiene campo eléctrico en la dirección de propagación.

    \item[TM]\mbox{}\\
    \textbf{Transversal Magnético.} Modo asociado a una onda electromagnética la cual no tiene campo magnético en la dirección de propagación.

    \item[UTD]\mbox{}\\
    \textbf{Uniform Geometrical Theory of Diffraction.} Teoría Geométrica Uniforme de Difracción. Técnica utilizada por los modelos numéricos en el estudio de las propiedades electromagnéticas de distintos cuerpos.

    \item[W]\mbox{}\\
    \textbf{Watio.} Unidad de potencia del Sistema Internacional.

    \item[Wi-Fi]\mbox{}\\
    \textbf{WideBand Fidelity.} Certificación que llevan muchos productos que trabajan con el estándar IEEE 802.11 a la frecuencia de 2.4 GHz.

\end{description}
