\chapter*{\textbf{Resumen}}

Los dispositivos médicos implantables (DMI) son aparatos que se colocan de manera parcial o total en el interior del cuerpo humano con el propósito de corregir disfunciones de órganos o monitorizar parámetros fisiológicos. Debido a su ubicación dentro del cuerpo humano, los DMI suelen ir equipados con un sistema de comunicaciones que permite configurar sus parámetros y recuperar datos monitorizados desde un equipo externo.

Los primeros sistemas de comunicaciones de los DMI hacían uso de tecnologías inductivas para crear un canal físico. Las limitaciones de las tecnologías inductivas, sumadas a los continuos avances realizados en materiales y componentes electrónicos, han propiciado en los últimos años la aparición de nuevas tecnologías basadas en comunicaciones por radiofrecuencia. En estas nuevas tecnologías, el diseño de la antena del sistema de comunicaciones del DMI supone un reto importante, puesto que debe satisfacer ciertos requisitos: debe ser pequeña, eficiente y de bajo consumo. Las \textit{antenas microstrip} poseen unas características que las hacen apropiadas para el sistema de comunicaciones de un DMI. Existen muchos diseños diferentes de estas antenas ya que no hay un modelo concreto de diseño o un único modelo de DMI.

Este Proyecto Fin de Carrera consiste en analizar, caracterizar y diseñar una antena microstrip en serpentina a partir de un modelo base para DMI extraído de un artículo de la literatura revisada para este proyecto. A través de un estudio paramétrico y simulaciones realizadas por computadora, se extraen resultados de distintos parámetros característicos en forma de gráficas e ilustraciones. Los resultados de las simulaciones de la antena diseñada indican que esta antena podría ser utilizada en un sistema de comunicaciones de un DMI.

